\documentclass[a4paper, 16pt]{article}
\usepackage[utf8]{inputenc}
\usepackage[english, russian]{babel} 
\usepackage[left=20mm, top=20mm, right=20mm,
 bottom=20mm, head=1mm, foot=1mm]{geometry}
\usepackage{tikz} 
\usepackage{amsmath, amsfonts, amssymb}
\usepackage{graphicx}
\usepackage{fancybox, fancyhdr}
\usepackage{hyperref}
\usepackage{listings}
\usepackage{caption}
\usepackage{xcolor}
\pagestyle{fancy}
\fancyhf{}
\fancyhead[L]{Лабораторная работа №1}
\fancyhead[R]{Программирование промышленных контроллеров}
\fancyfoot[C]{\thepage}
\graphicspath{{images/}}
\usetikzlibrary{patterns}
\hypersetup{
    colorlinks=true,
    linkcolor=blue,
    filecolor=magenta,      
    urlcolor=cyan,
    pdftitle={contents setup},
    pdfpagemode=FullScreen,
}
\newcommand{\frc}[2]{\raisebox{2pt}{$#1$}\big/\raisebox{-3pt}{$#2$}}

\begin{document}
    \begin{titlepage}
        \begin{center}
        \vfill
        
        Федеральное государственное автономное образовательное учреждение высшего образования\\
        «Национальный Исследовательский Университет ИТМО»\ \\
        
        \vfill
        {\large\bf ЛАБОРАТОРНАЯ РАБОТА №1\\
            ПО ПРЕДМЕТУ «ПРОГРАММИРОВАНИЕ ПРОМЫШЛЕННЫХ КОНТРОЛЛЕРОВ»}
        \vfill
            
        \begin{flushright}
            \begin{minipage}{.45\textwidth}
            {
                \hbox{Преподаватель: Крылова А. А.}
                \hbox{Выполнил: Румянцев А. А.}
                \hbox{}
                \hbox{Факультет: СУиР}
                \hbox{Поток: ПРОГ. ПРОМ.ЛК 2.2}
            }
            \end{minipage}
        \end{flushright}
        
        \vfill
                
        Санкт-Петербург\\
        2024
        \end{center}
    \end{titlepage}
    \setlength{\parskip}{1.5mm}
    
    \tableofcontents

    \newpage

\end{document}